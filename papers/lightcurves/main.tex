% mnras_template.tex
%
% LaTeX template for creating an MNRAS paper
%
% v3.0 released 14 May 2015
% (version numbers match those of mnras.cls)
%
% Copyright (C) Royal Astronomical Society 2015
% Authors:
% Keith T. Smith (Royal Astronomical Society)

% Change log
%
% v3.0 May 2015
%    Renamed to match the new package name
%    Version number matches mnras.cls
%    A few minor tweaks to wording
% v1.0 September 2013
%    Beta testing only - never publicly released
%    First version: a simple (ish) template for creating an MNRAS paper

%%%%%%%%%%%%%%%%%%%%%%%%%%%%%%%%%%%%%%%%%%%%%%%%%%
% Basic setup. Most papers should leave these options alone.
\documentclass[a4paper,fleqn,usenatbib]{mnras}

% MNRAS is set in Times font. If you don't have this installed (most LaTeX
% installations will be fine) or prefer the old Computer Modern fonts, comment
% out the following line
%\usepackage{newtxtext,newtxmath}
\usepackage{import}
\usepackage{hyperref}

% Depending on your LaTeX fonts installation, you might get better results with one of these:
%\usepackage{mathptmx}
%\usepackage{txfonts}

% Use vector fonts, so it zooms properly in on-screen viewing software
% Don't change these lines unless you know what you are doing
\usepackage[T1]{fontenc}
\usepackage{ae,aecompl}

\usepackage{graphicx}	% Including figure files
\usepackage{amsmath}	% Advanced maths commands
\usepackage{amssymb}	% Extra maths symbols
\usepackage{pdflscape}	% Landscape pages
\usepackage[utf8]{inputenc}
\usepackage{multirow}
\usepackage{float} % here for H placement parameter

%%%%%%%%%%%%%%%%%%%%%%%%%%%%%%%%%%%%%%%%%%%%%%%%%%

%%%%% AUTHORS - PLACE YOUR OWN COMMANDS HERE %%%%%

% Please keep new commands to a minimum, and use \newcommand not \def to avoid
% overwriting existing commands. Example:
%\newcommand{\pcm}{\,cm$^{-2}$}	% per cm-squared

%%%%%%%%%%%%%%%%%%%%%%%%%%%%%%%%%%%%%%%%%%%%%%%%%%

%%%%%%%%%%%%%%%%%%% TITLE PAGE %%%%%%%%%%%%%%%%%%%

% Title of the paper, and the short title which is used in the headers.
% Keep the title short and informative.
\title[A reference transient dataset I: lightcurves]{A reference
  dataset for astronomical transient event recognition I: lightcurves
  and tests on classical machine learning algorithms}


% The list of authors, and the short list which is used in the headers.
% If you need two or more lines of authors, add an extra line using \newauthor
\author[M. Neira et al.]
{Mauricio Neira$^{1}$, Catalina G\'omez$^{2}$, Diego A. G\'omez $^{1}$,
Juan Pablo Reyes$^{1}$,
\newauthor
Marcela Hern\'andez Hoyos$^{1}$,   
Pablo Arbel\'aez$^{2}$,
Jaime E. Forero-Romero$^{3}$
\\
% List of institutions
$^{1}$Systems and Computing Engineering Department, Universidad de los Andes, Cra. 1 No. 18A-10, Bogot\'a, Colombia\\
$^{2}$Departamento de Ingenier\'ia Biom\'edica, Universidad de los Andes, Cra. 1 No. 18A-10, Bogot\'a, Colombia\\
$^{3}$Departamento de F\'isica, Universidad de los Andes, Cra. 1 No. 18A-10, Bogot\'a, Colombia
}

% These dates will be filled out by the publisher
\date{Accepted XXX. Received YYY; in original form ZZZ}

% Enter the current year, for the copyright statements etc.
\pubyear{2019}

% Don't change these lines
\begin{document}
\label{firstpage}
\pagerange{\pageref{firstpage}--\pageref{lastpage}}
\maketitle

% Abstract of the paper
\begin{abstract}

We introduce ATRANCCATA (Annotated TRANsient Catalina CATAlog) an
annotated dataset of $4869$ transient and $16940$ non-transient
object lightcurves built from the Catalina Real Time Transient
Survey.
We provide public access to this dataset as a plain text file to facilitate
standardized quantitative comparison of astronomical transient event
recognition algorithms. 
Some of the classes included in the dataset are: supernovae, cataclismic
variables, active galactic nuclei, high proper motion stars, blazars
and flares.
As a complement to the dataset, we experiment with multiple
data pre-processing methods, feature selection techniques and popular
machine learning algorithms (Support Vector Machines, Random Forests
and Neural Networks).   
We assess quantitative performance in two classification tasks:
binary (transient/non-transient) and eight-class classification.   
The best performing algorithm is a Random Forest Classifier for both
classification experiments.  
The next release of ATRANCCATA will include images and benchmarks with
deep learning models. 
All our code and data is available to the community at
\url{https://github.com/MachineLearningUniandes/ATRANCCATA}.
\end{abstract}

% Select between one and six entries from the list of approved keywords.
% Don't make up new ones.
\begin{keywords}
methods: data analysis, statistical
%keyword1 -- keyword2 -- keyword3
\end{keywords}

%%%%%%%%%%%%%%%%%%%%%%%%%%%%%%%%%%%%%%%%%%%%%%%%%%

%%%%%%%%%%%%%%%%% BODY OF PAPER %%%%%%%%%%%%%%%%%%

\section{Introduction}

The study and detection of astronomical variable sources is expected
to occur at unprecedented scales with the new generation of
forthcoming multi-epoch and multi-band (synoptic) astronomical
surveys. 
For instance, projects like the Large Synoptic Survey Telescope
(LSST)  \citep{0805.2366,1512.07914} are expected to generate
exuberant daily data-streams of about 20 petabytes every night.  

One of the main challenges that these datasets want to address is
Real-Time Transient classification \citep{2004SPIE.5489...11K,2009PASP..121.1395L,2014ApJ...788...48S,2018PASP..130f4505T,2019PASP..131a8002B} including phenomena such
as supernovae (SN), novae, neutron 
stars, blazars, pulsars, cataclysmic variable stars (CV), gamma ray 
bursts (GRB) and active galaxy nuclei (AGN). 
The time-domain dependency of these objects is one of the reasons why 
they are hard to classify: their data is usually heterogeneous, 
unbalanced, sparse, unevenly sampled and with missing information. 
These challenges have motivated the use of Machine Learning (ML) algorithms to recognize and classify transient events. 



There have been successful attempts to implement these algorithms
using images as input. 
For instance, data from the SkyMapper Supernova and Transient 
Survey and the High cadence Transient Survey (HiTS) have been used as 
input to automatic detection algorithms \citep{1708.08947,1701.00458}. 
Convolutional Neural Networks (CNN) have also achieved 
high accuracy in this binary classification task.
Other studies have shown that artifacts can be detected using
features extracted from raw images. 
\cite{1601.06320} achieved reliable
classification by transforming transient data from the OGLE-IV
data-reduction pipeline and training it with machine learning
algorithms such as Artificial Neural Networks, Support Vector
Machines, Random Forests, Naive Bayes, K-Nearest Neighbors and Linear
Discriminant Analysis.  
Similarly, \cite{1501.05470} used images from Pan-STARS1 Medium Deep
Surveys, and \cite{1407.4118} processed single-epoch multi-band images
from the SDSS supernova survey for the same purpose.  


A complementary approach uses the lightcurves computed from the
images to perform the classification task.
For instance \cite{1601.03931} used classical machine learning
algorithms such as Random Forest, MultiLayer Perceptron and K-Nearest Neighbours
lightcurves to classify transients from the Catalina Real Time
Transient Survey; \cite{1603.00882} used the same approach to find
supernovas from the Supernova Photometric Classification
Challenge.

A crucial element in the development of any ML algorithm is the
compilation of the training dataset. 
In astronomy this task has been facilitated by the publication of
large databases from different observational projects. 
However, the publications that make use of these datasets still make
extensive use of the internal know-how of the scientific
collaboration. 
It is still difficult for other scientists to rebuild a training
dataset and perform comparisons with published results.

To address this issue, we compile and publish in easy-to-access files a
dataset that can be used to train and test different ML algorithms for
transient detection.
We use public data from the Catalina Real-Time Transient Survey
(CRTS) \citep{1111.2566}, an astronomical survey searching transient
and highly variable objects as base for the dataset.
In this paper we present lightcurve data, in a future publication we
will present an imaging dataset. 

In Section \ref{sec:data} we present the CRTS and the steps we follow
to build the dataset.
Then, in Section \ref{sec:repository} we describe its main features together
with the repository structure gathering the files and Python code to explore it. 
In Section \ref{sec:ml_tests} we show how this dataset can be used to 
perform tests using ML methods following a similar approach as \cite{1601.03931},
and the experiments that we perform. 
We finalize in Section \ref{sec:conclusions} with a summary of the
main features of our dataset and the results of our experiments. 


\section{The lightcurve dataset} 
\label{sec:data}

We use public data from the Catalina Real-Time Transient Survey
(CRTS) \citep{2009ApJ...696..870D}, an astronomical survey searching transient
and highly variable objects.
The CRTS covered 33000 squared degrees of sky and took data since 2007. 
Three telescopes were used: Mt. Lemmon Survey (MLS), Catalina Sky 
Survey (CSS), and Siding Spring Survey (SSS). So far, CRTS has 
discovered more than $15000$ transient events.
We use data from the CSS telescope, which is an f/1.8 Schmidt
telescope located in the Santa Catalina Mountains in Arizona.
The telescope is equipped with a 111-megapixel  detector, and covered
4000 square degrees per night, with a limiting magnitude of 19.5 in
the V band.  

Putting together the lightcurves for ATRANCCATA implies
cross-matching different files in the legacy CRTS webpage:
\url{http://nesssi.cacr.caltech.edu/DataRelease/CRTS-I_transients.html}. 
The photometry is stored in two different kinds of files: \verb"phot"
that come from the main photometry database and \verb"orphan" that
correspond to transients not associated with the 500 million sources
in the main photometry database.
There are also \verb"out" files that must be used to link transient
IDs to database IDs.


For each one of the 5540 transients reported and classified in the
archival webpage \url{http://nesssi.cacr.caltech.edu/catalina/All.arch.html} we use its
transient IDs and its database IDs to look for the lightcurves in the
\texttt{phot} and \texttt{orphan} files. 
Only 4982 transients can be linked to available data to reconstruct
their lightcurves. 
Furthermore, some of these lightcurves are duplicated, i.e. they had
the same number of observations, Modified Julian Date (MJD) and magnitude measurement. 
We ignore the duplicates to end up with 4869 unique transients with an 
associated lightcurve. Figure \ref{fig:transients} summarizes this process. 

\begin{figure*}
	\includegraphics[width=0.9\textwidth]{Transients.pdf}
  \caption{ATRANCCATA Dataset Set Up: Lightcurve compilation for transient classes.}
  \label{fig:transients}
\end{figure*} 

The CRTS dataset already provides a classification. 
The most numerous classes are: supernovae,
cataclysmic variable stars, blazars, flares, asteroids, active
galactic nuclei, and high-proper-motion stars (HPM). 
Though most objects in the transient object catalogue belong to a single class, 
there is some uncertainty in the categorization of some of them.  
In this case, an interrogation sign is used when a class is not clear 
e.g. SN? or sometimes multiple possible classes are found for a single 
event e.g. SN/CV. 
Table \ref{table:top_classes} summarizes the number of objects in each class. 

To compile the non-transient lightcurves we retrieve sources in the
dataset from the CRTS online catalogue by retrieving objects within a
20 arcsecond radius from CRTS detected transients, 
and removing known transient lightcurves from that set. 
We end up with 16940 unique non-transient lightcurves. Figure \ref{fig:non-transients} illustrates this process.

\begin{figure*}
	\includegraphics[width=0.7\textwidth]{NonTransients.pdf}
  \caption{ATRANCCATA Dataset Set Up: Lightcurve compilation for non-transients.}
  \label{fig:non-transients}
\end{figure*} 

% Number of transients per transient class
\begin{table}
\centering
\begin{tabular}{c|c}
    \hline
    Class &  Object Count \\
    \hline
SN & 1723 \\
CV & 988 \\
HPM & 640 \\
AGN & 446 \\
SN? & 319 \\
Blazar & 243 \\
Unknown & 228 \\
Flare & 219 \\
AGN? & 138 \\
CV? & 77 \\
    \hline
\end{tabular}
\caption{Top 10 transient classes in the CRTS with their respective number of lightcurves.} 
\label{table:top_classes}
\end{table}


\begin{figure*}
	\includegraphics[width=0.45\textwidth]{cumulative_magnitude.pdf}
  \includegraphics[width=0.45\textwidth]{cumulative_classes.pdf}
  \caption{Cumulative number of lightcurves (expressed as a fraction)
    as a function of average magnitude (left) and number of data
    points in the lightcurve (right).
    This includes information for the three most representative
    classes (SN, CV, AGN) and the whole database (ALL).}
  \label{fig:cumulative}
\end{figure*} 


\begin{figure*}
  \includegraphics[width=0.6\textwidth]{examples_transient.pdf}
  \caption{Randomly selected lightcurves for the most represented transient classes as compiled in ATRANCCATA. The class of each sample is within the legend box. }  
  \label{fig:examples_transient}
\end{figure*} 


\begin{figure*}
  \includegraphics[width=0.6\textwidth]{examples_nontransient.pdf}
  \caption{Randomly selected lightcurves for non-transient sources retrieved for ATRANCCATA.}
  \label{fig:examples_non_transient}
\end{figure*} 

Figure \ref{fig:cumulative} shows the number of lightcurves as a
function of average magnitude (left panel) and as a function of the
number of points in the lightcurve (right panel).
We show separately the whole data set and three most representative
classes: supernova, cataclysmic variables and active galactic nuclei. 
For these four sets, the median magnitude is in the range $18-20$. 
The number of points in the lightcurve has a larger variability.
The median for all the curves is close to 30, while for SN, CV and
AGN it is close to 15, 50 and 180, respectively. 
We provide sample lightcurves of the most represented transient classes 
and non-transient sources in Figure \ref{fig:examples_transient} and Figure \ref{fig:examples_non_transient}, respectively. 
The brightness evolution of non-transient sources is more stable over time, 
while transient objects present non-periodical changes at different time scales. 

%resatar dificultad del problema: no hay una base de datos estándar con datos reales
The challenges in the classification of lightcurves 
include the inherent nature of transient events, which is reflected 
in different brightness behaviors, their evolution over time, and 
the nonuniform sampling of observations at sequential dates. 
Besides, there is a large class imbalance to localize transient events, 
and perform their subsequent classification. 


\subsection{Classification Tasks} \label{subsection_classification}
We study two classification tasks on the ATRANCCATA dataset: 

\begin{itemize}
\item {Binary Classification}.
Using a balanced number of events from both classes in order 
to investigate the capability of distinguishing between Transient
and non-transient sources.
\item{8-Class Classification}.
Using the unbalanced number of objects across classes to 
perform a classification into the following categories:
AGN, Blazar, CV, Flare, HPM, Other, SN and Non-Transient.
\end{itemize}

We evaluate both tasks using the metrics of a detection problem. 
For each class in the testing set, we report the maximum F1-Score 
that is defined as the harmonic mean of precision and recall. 
We construct Precision-Recall (PR) curves by setting different 
thresholds on the output probabilities of belonging to each class. 


\section{Repository Description} 
\label{sec:repository}

The repository contains the lightcurves and a Jupyter notebook
to reproduce some of the Figures and Tables in this paper.
The repository can be found in \url{https://github.com/MachineLearningUniandes/ATRANCCATA}. 
To date the repository has two main folders:
\begin{itemize}

\item \texttt{data/lightcurves}: 
contains three text files in CSV format
the transient lightcurves (\texttt{transient\_lightcurves.csv}),
the labels for the transients (\texttt{transient\_labels.csv}) and
the lightcurves for non-transient objects
(\texttt{nontransient\_lightcurves.csv}). 
The first two files can be linked by unique transient IDs and
provided in the CRTS database. 
\item \texttt{nb-explore}: includes a jupyter notebook
  (\texttt{explore\_light\_curves.ipynb}) with examples on how to read
  and plot transient and non-transient lightcurves, extract the statistics in Table
  \ref{table:top_classes} and prepare the summary statistics in Figure
  \ref{fig:cumulative}. 
  Additional python files (\texttt{features.py},
  \texttt{helpers.py} and \texttt{inputs.py}) allow to read and perform
  simple operations on the CSV data files. 
\end{itemize}



\section{Machine Learning Methods} 
\label{sec:ml_tests}
In order to provide baseline algorithms on the ATRANCCATA dataset 
that can be used as a reference for future work, we apply ML methods 
to perform different classification tasks. 
In Figure \ref{fig:ML} we show an overview of our method for transient classification. 
The main steps include data processing, feature extraction and classification.  


\subsection{Preprocessing and Feature Extraction}
We do not input directly the annotated lightcurves to the ML algorithms.
We perform a preprocessing stage as follows. 
First, we discard lightcurves with less than 10 data points observations
as they may not contain enough information to be classified correctly.

\begin{figure*}
	\includegraphics[width=0.9\textwidth]{ML.pdf}
  \caption{Overview of the Machine Learning process on the ATRANCCATA dataset for the binary and 8-class classification tasks. We take the raw lightcurves as input, preprocess the data (step 1) and balance the classes for the training phase (step 2). The final classification (step 4) follows the feature extraction stage (step 3).}
  \label{fig:ML}
\end{figure*} 


Given that the number of lightcurves per class is imbalanced, 
in order to have the same number of instances for each class, we implement an
oversampling step by artificially generating multiple mock lightcurves 
from an observed one. 
We generate a slightly different lightcurve from the observed lightcurve and 
then sample the observed magnitude from a Gaussian distribution
centered on the observational apparent magnitude with the magnitude's
error as the standard deviation. 


Finally, we compute a standard set of features for each lightcurve. 
These features are scalars derived from statistical and model-specific
fitting techniques. 
The first features (moment-based, magnitude-based and
percentile-based) were formally introduced in 
\cite{1101.1959}, and have been used in other studies \citep{1603.00882,1601.03931}.  
We extend that list to include another set (polynomial fitting-based features). 
At the end of this process we normalize the features to have zero
mean and unit variance.  

These groups of features are:

\begin{enumerate}
    
\item Moment-based features, which use the magnitude for each lightcurve.
  \begin{itemize}
  \item \texttt{beyond1std}: 
    Percentage of observations which are over or under one standard
    deviation from the weighted average. Each weight is calculated as
    the inverse of the corresponding observation's photometric error. 
  \item \texttt{kurtosis}: 
    The fourth moment of the data distribution. 
  \item \texttt{skew}: 
    Skewness. Third moment of the data distribution.
  \item \texttt{sk}:
    Small sample kurtosis.
  \item \texttt{std}::
    The standard deviation.
  \item \texttt{stetson\_j}:
    The Welch-Stetson J variability index
    \citep{1996PASP..108..851S}. A robust standard deviation. 
  \item \texttt{stetson\_k}:  The Welch-Stetson K variability index
    \citep{1996PASP..108..851S}. A robust kurtosis measure. 
  \end{itemize}
  
\item Features based on the magnitudes.
    \begin{itemize}
    \item \texttt{amp}: 
      The difference between the maximum and minimum magnitudes.
    \item \texttt{max\_slope}: 
      Maximum absolute slope between two consecutive observations.
    \item \texttt{mad}: 
      The median of the difference between magnitudes and the median
      magnitude. 
    \item \texttt{mbrp}: 
      The percentage of points within 10\% of the median magnitude.
    \item \texttt{pst}: 
      Percentage of all pairs of consecutive magnitude measurements that have positive slope.
    \item \texttt{pst\_last30}: 
      Percentage of the last 30 pairs of consecutive magnitudes that
      have a positive slope, minus percentage of the last 30 pairs of
      consecutive magnitudes with a negative slope. 
    \end{itemize} 


  \item Percentile-based features, which use the sorted flux distribution for
    each source. The flux is computed as $F = 10^{0.4 \mathrm{mag}}$. 
    We define $F_{n,m}$ as the difference between the $m$-th and $n$-the flux
    percentiles. 
    \begin{itemize}
    \item \texttt{p\_amp}: 
      Largest percentage difference between the absolute maximum
      magnitude and the median. 
    \item \texttt{pdfp}: 
      Ratio between $F_{5,95}$ and the median flux.
    \item \texttt{fpr20}: 
      Ratio $F_{40,60} / F_{5,95}$
    \item \texttt{fpr35}:
      Ratio $F_{32.5,67.5} / F_{5,95}$
    \item \texttt{fpr50}: 
      Ratio $F_{25,75} / F_{5,95}$
    \item \texttt{fpr65}: 
      Ratio $F_{17.5,82.5} / F_{5,95}$
    \item \texttt{fpr80}: 
      Ratio $F_{10,90} / F_{5,95}$
    \end{itemize}
    
  \item Polynomial Fitting-based features, which are the coefficients of
    multi-level terms in a polynomial curve fitting. This is a new set
    of features proposed in this paper. 
    \texttt{Polyn\_Tm} indicates the coefficient of the term of order
    \texttt{m} in a fit to a polynomial of order \texttt{n}.
    \begin{itemize}
        \item \texttt{Poly1\_T1}.
        \item \texttt{Poly2\_T1}.
        \item \texttt{Poly2\_T2}.
        \item \texttt{Poly3\_T1}.
        \item \texttt{Poly3\_T2}.
        \item \texttt{Poly3\_T3}.
        \item \texttt{Poly4\_T1}.
        \item \texttt{Poly4\_T2}.
        \item \texttt{Poly4\_T3}.
        \item \texttt{Poly4\_T4}.
    \end{itemize}    
\end{enumerate}


\subsection{ML algorithms}

We conduct experiments with three widely used families of supervised classification 
algorithms \citep{skysurvey, 1601.03931}: Neural Networks (NNs), Random Forests (RFs) and Support
Vector Machines (SVMs). 

These algorithms are popular in published studies and are efficient 
for low dimensional feature datasets as is our case. 
We use sklearn \citep{1201.0490} Python's implementation of these algorithms.
Details on the inner workings of these machine learning models can be
found in \cite{9780387848570}.  
The hyperparameters explored for each algorithm are the
following. 

\begin{itemize}
\item Neural Networks:
\begin{itemize}
\item Learning Rate: Either constant vs. adaptive.
\item Hidden Layer Sizes: Single Layer with $100$ nodes vs. Two layers with
  $100$ nodes each.
\item L2 Penalty ($\alpha$): 
  $10^{-1}$, $10^{-2}$, $10^{-3}$ , $10^{-4}$.
\item Activation Function: Logistic vs. Relu.
\end{itemize}

\item Random Forest:
\begin{itemize}
    \item Number of Estimators: $200$ or $700$.
    \item Number of features considered: Square Root or the Logarithm 
    base 2 of the total number of features.
\end{itemize}

\item Support Vector Machines:
\begin{itemize}
    \item Kernel: Radial Basis Function (RBF).
    \item Kernel Coefficient ($\gamma$):  
      $10^{-1}$, $10^{-2}$, $10^{-3}$ , $10^{-4}$, $10^{-5}$
    \item Error Penalty (\textit{C}): 1 vs 10 vs 100 vs 1000. 
\end{itemize}
\end{itemize}

\subsection{Validation} \label{subsection_importances}

We split the input lightcurves into training and testing. 
The testing set contains 4869 unique transient lightcurves.
We use 2-fold cross-validation during training as evaluation
protocol. 
Moreover, we use grid search during training to test multiple 
hyperparameter configurations for each one of the proposed methods.


\begin{figure*}
	\includegraphics[width=\textwidth]{binFeatImportance.pdf}
    \caption{Feature importance rank  for the best Random Forest
      classifier for the Binary classification task. 
      Feature importance is represented with percentages.} 
    \label{Importances-Binary}
\end{figure*} 

\begin{figure*}
	\includegraphics[width=\textwidth]{8classFeatImportance.pdf}
    \caption{Feature importance rank for the best Random Forest
      classifier for the best 8-Class classification task. Feature
      importance is represented with percentages.} 
    \label{Importances-8-Class}
\end{figure*}

\subsection{Results}

%%%%%%  BINARY  %%%%%%
\subsubsection{Binary Classification} 
\label{Results-Binary} 

The best algorithm in this task is RFs with an average F1-Score of
87.69\%.   
SVMs are the second best-performing model with a F1-Score of 85.36\%. 
Changing the number of features does not significantly affect the score.
NNs are ranked third, although their scores are very similar to those of SVMs. 
The highest achieved score for NNs is 85.03\%.

Table \ref{Confusion-Binary} shows the confusion matrix of the best
performing algorithm and Table \ref{Overall-Scores-Binary} summarizes 
the metrics for transient and non-transient class. These results suggest 
that in an imbalanced set up, non-transient sources are better classified, 
while transients are more difficult, showing a difference of about 
14 points in F1-Score compared to the non-transient class. 
This difference in performance could be attributed to the intra-class
variation within the transient class because of the different 
types of transient sources.  

Figure \ref{Importances-Binary} displays the most important features
for the RFs classifier.
The top five inputs for classification are \texttt{stetson\_j},
\texttt{std}, \texttt{mad}, \texttt{poly1\_t1} and \texttt{poly2\_t1}.  
The first feature achieved the highest importance of 21\%, compared to
the following with values in the range 6\% - 8\%.  


\begin{table}
\centering
\begin{tabular}{|r|c|c|c|c|}
\hline
\multicolumn{1}{|l|}{} & Precision & Recall & F1-Score & No. instances \\ \hline \hline
Non-Transient          & 94.13     & 94.13      & 94.13      & 3798   \\ \hline
Transient              & 79.10     & 79.10      & 79.10      & 1067    \\ \hline
\end{tabular}
\caption{Precision, Recall and F1-Score for the Binary Classification Task.}
\label{Overall-Scores-Binary}
\end{table}

\begin{table}
\centering
\begin{tabular}{|r|c|c|}
\hline
\multicolumn{1}{|l|}{} & Non-Transient    & Transient   \\ \hline \hline
Non-Transient                & 3575       & 223    \\ \hline
Transient                    & 223       & 844   \\ \hline
\end{tabular}
\caption{Confusion Matrix for the best performing model in the Binary task. Rows represent prediction and columns the ground truth.}
\label{Confusion-Binary}
\end{table}



%%%%%%  EIGHT-CLASS  %%%%%%
\subsubsection{Eight-Class Classification}

For this task, RFs are again the best classifier.  
The best F1-Score is 66.05\%. 
NNs are the second best model. 
Their highest F1-Score is 60.19\%, while SVMs are the worst-performing
model only achieving an average F1-Score of 57.30\%.
Table \ref{Overall-Scores-8-Class-Regular} summarizes the results for 
individual classes and Table \ref{Confusion-8-Class} presents 
the confusion matrix for the RFs.

The two classes with highest F1-Score are non-transient (87.12\%) and CV (68.77\%). 
The recall decreases for the non-transient class in comparison to the binary experiment, 
meaning that the algorithm misclassified some instances that belong to non-transient class
among transient classes. 
However, transient sources are not commonly confused with non-transient ones. 
The worst performing classes are Flare, Other and HPM, with F1-Scores in the 
range 11\% - 40\%. 
It is worth noting that the less frequent classes present a lower performance, 
such as Flare and HPM. 
Even though the most frequent classes are more easily identified, 
the "other" type class has a low F1-score due to the different nature 
of sources that were assigned to this category. 


SN is the class with which most other class instances are
incorrectly classified. 
Moreover, Flares have about 50\% of the test samples classified as
non-transients, AGNs have about 20\% of their 
samples classified as Other, and Blazars and Other had most of  its
samples classified as AGN. 
Additionally, most incorrectly classified AGNs ($\sim$20.5\%) are
identified as Other, and most Blazar instances are
incorrectly categorized as either SN or AGN. 


Figure \ref{Importances-8-Class} displays the feature importance ranking.
This list ranks first \texttt{stetson\_j} with an 8\% importance,
followed by \texttt{amp}, \texttt{sk}, \texttt{std}, \texttt{mad},
with values around 6\%.   


\begin{table}
\centering
\begin{tabular}{|r|c|c|c|c|}
\hline
\multicolumn{1}{|l|}{} & Precision & Recall & F1-Score & No. instances \\ \hline \hline
SN            &   48.82 &   51.39  &  50.07  &  323 \\ \hline
CV            &   66.96 &   70.69  &  68.77  &  215 \\ \hline
AGN           &   48.14 &   85.84  &  61.69  &  106 \\ \hline
HPM           &   25.19 &   86.84  &  39.05  &   76 \\ \hline
Blazar       &   46.77 &   49.15  &  47.93  &   59 \\ \hline
Flare       &    7.00 &   41.17  &  11.96  &   51 \\ \hline
Other         &   31.11 &   44.01  &  36.46  &  234 \\ \hline
Non-Transient &   96.06 &   79.69  &  87.12  & 3798 \\ \hline
avg/total     &   46.25 &   63.59  &  50.38  & 4862 \\ \hline
\end{tabular}
\caption{Precision, Recall and F1-Score for the 8-Class Classification Task.}
\label{Overall-Scores-8-Class-Regular}
\end{table}

\begin{table}
\centering
\begin{tabular}{|r|c|c|c|c|c|c|c|c|}
\hline
\multicolumn{1}{|l|}{} & \rotatebox{90}{SN}    & \rotatebox{90}{CV}
& \rotatebox{90}{AGN}   & \rotatebox{90}{HPM}    &
\rotatebox{90}{Blazar}  & \rotatebox{90}{Flare}  &
\rotatebox{90}{Other}   & \rotatebox{90}{Non-Tr.}  \\ \hline \hline
SN            & 166  &  25  &  0  &  0  &  7   &  5  &  40 &   97 \\ \hline
CV            &  17  & 152  &  0  &  1  &  5   &  3  &  12 &   37 \\ \hline
AGN           &   1  &   2  & 91  &  0  & 10   &  1  &  35 &   49 \\ \hline
HPM           &   5  &   0  &  0  & 66  &  0   &  0  &   5 &  186 \\ \hline
Blazar       &   8  &  13  &  4  &  0  & 29   &  0  &   6 &    2 \\ \hline
Flare        &  16  &   5  &  0  &  0  &  3   & 21  &   4 &  251 \\ \hline
Other        &  53  &  12  &  7  &  1  &  3   &  3  & 103 &  149 \\ \hline
Non-Tr. &  57  &   6  &  4  &  8  &  2   & 18  &  29 & 3027 \\ \hline
\end{tabular}
\caption{Confusion Matrix for the best performing model in the 8-class
  task. The classes follow the abbreviations in Table \ref{Overall-Scores-8-Class-Regular}. Rows represent the predictions, and columns the ground truth.}
\label{Confusion-8-Class}
\end{table}



\section{Conclusions}
\label{sec:conclusions}

The scope of forthcoming large astronomical synoptic surveys 
motivates the development and exploration of automatized ways to
detect transient sources. 
In turn, this need prompts the compilation of publicly available databases
to train and test new algorithms. 
In this paper we presented the results of such a compilation based on data
from the Catalina Real-Time Transient Survey.
The data-set compiles  $4869$ transient and $16940$ non-transient
lightcurves. 
The dataset is publicly available at
\url{https://github.com/MachineLearningUniandes/ATRANCCATA}.   

We illustrated how to use this database by extracting 
characteristic features to use them as input to train three different
machine learning algorithms (Random Forests, Neural Networks and
Support Vector Machines) for classification tasks.
The features extracted from lightcurves were either statistical
descriptors of the observations, or polynomial curve fitting
coefficients applied to the lightcurves.   
Overall, the best classifier for all tasks was the Random Forest.
In this model the most important feature was always
\texttt{stetson$\_$j}, i.e. a robust estimate for the standard
deviation. 

In a second paper we will present another reference dataset for
astronomical transient event recognition based on images of the
CRTS.
The corresponding tests will use  state-of-the art deep learning
techniques for transient classification. 

\section*{Acknowledgements}

We thank Andrew Drake for sharing with us the CRTS Transient dataset
used in this project.  
We acknowledge funding from Universidad de los Andes in the call for
project finalization.
We also thank contributors and collaborators of the SciKit-Learn,
Jupiter Notebooks and Pandas' Python libraries.  

CRTS and CSDR2 are supported by the U.S.~National Science 
Foundation under grant NSF grants AST-1313422, AST-1413600, and 
AST-1518308.  The CSS survey is funded by the National Aeronautics
and Space Administration under Grant No. NNG05GF22G issued through
the Science Mission Directorate Near-Earth Objects Observations Program.

\bibliographystyle{mnras}
\bibliography{bibliography}

\end{document}

